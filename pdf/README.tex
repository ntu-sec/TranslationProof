\documentclass[a4]{article}

\usepackage{hyperref}
\usepackage{algorithm}
\usepackage{algorithmic}
\usepackage{MnSymbol}

\begin{document}

\section{Infrastructure}
Here are the tools that we used to produce the prototype and the demo videos.
\begin{description}
  \item[Android Phone] The phone that we use for the experimentation is 
    Samsung Galaxy 4, model number of GT-I9506, and Android version 5.0.1
  \item[Android Studio] The tool we used to develop the prototype of an application
    that can do the type checking and a test app. You can get it from 
    \href{http://developer.android.com/develop/index.html}
    {http://developer.android.com/develop/index.html}.
  \item[OCaml] The tool we used to develop the prototype of certificate translation.
    You can get it from \href{http://ocaml.org}{http://ocaml.org}
  \item[Vysor] A google chrome app to show the screen of an Android phone on the 
    computer screen.
\end{description}

\section{Instructions for Constructing Certificate}
At this stage, since we only have the translation from JVM classes to DEX, we need to 
get the classes from a half compiled Android app. Using the Android Studio we can do
this by first building the non-optimized app, then get the classes from the 
intermediates directory. After we get the classes we feed these classes into the OCaml
program to get the certificate which contains the typing for the Android app. Finally
we rename the certificate into ``Certificate.cert'' and insert it into the ``assets/''
directory, and compile the application.

\section{Instructions for Compilation}
\subsection{APK Reader}
We developed this APK Reader using the Android Studio tool, which you can get from
\href{http://developer.android.com/develop/index.html}{http://developer.android.com/develop/index.html}. 
We have tested this app in our own commercial Android phone. 

\subsection{Test application}
We developed this test using the Android Studio tool. 
A note about this test application is that we have to put the option of no-optimize for
the DX compiler. This is a little bit complicated with the Android studio which uses
gradle version 1.5. One way to circumvent this problem is to build the app using
lower version of gradle (we used 1.3.1). To do this we can modify the graddle setting
for the project (contained in ``build.gradle'' at the root of the project directory), 
changing the value of classpath from ``com.android.tools.build:gradle:1.5.0'' to 
``com.android.tools.build:gradle:1.3.1''.
Then we modify the graddle setting for the module (contained in ``app/build.gradle'') .
We inject the following code into the gradle file:
\begin{algorithm}
\begin{algorithmic}
\STATE afterEvaluate \{
\STATE \hspace{0.5cm}tasks.matching \{ it.name.startsWith('dex') \}.each \{ dx $\rightarrow$
\STATE \hspace{0.5cm}if (dx.additionalParameters == null) \{ dx.additionalParameters = [] \}
\STATE \hspace{0.5cm}dx.additionalParameters += '--no-optimize' \}
\STATE \}
\end{algorithmic}
\end{algorithm}\\
We have tested this app in our own commercial Android phone. 

\subsection{Certificate Translation}
There is a makefile for the OCaml program already, so we just need to type ``make'' in
in the root directory. The output will be a binary called ``dx\_ocaml''.

\section{Instructions for Type Checking}
After opening the application, just provide the package name for the application that
you want to type check into the input text, and then click on the button. The application
will then parse the DEX file (which takes a really long time) and the certificate, then
provide the final judgement whether the application is type check or not.

\section{File Contents}
Apart from this file, the contents of this repositories are:
\begin{description}
  \item[Extended Paper.pdf] Contains the extended contents for the paper
  \item[files/simpleapp.apk] A simple test app which contains no method calling and    
    exception handling in the program. This app is compiled using the option of 
    no-optimize.
  \item[files/apkreader.apk] A prototype of the proof of concept. This app can read other
    app's APK and then check it against the certificate. The checked app can not contain
    method calling and exception handling. The app must also be compiled with the flag
    no-optimize.
  \item[files/SimpleApp/] The source for the test app.
  \item[files/APKReader/] The source code for the prototype. Two main contents of this
    app are the component to parse another file's apk, and the component to 
    type check it.
  \item[files/dx\_ocaml/] The source code for the OCaml files to provide a certificate
    for an app given its classes source. This component will first parse the JVM class,
    do a naive JVM type inference, then translate the certificate. 
  \item[files/Certificates/] Some sample certificates that match the test app. There are
    several certificates there which consists of a proper certificate, and other
    modified certificates which will trigger the type checking to fail. 
    There are also files which highlight why the type checking will fail.
    \begin{description}
      \item[Certificate.cert] This one is a proper certificate which will pass the type
        checker.
      \item[Certificate (Const rt).cert] This certificate will trigger a failure in the
        type checking because the registers typing for the Const instruction is 
        $\nsqsubseteq$ its successor's.
      \item[ConstRT\_typing] Show the typing for Const instruction and its successor's
      \item[Certificate (Invoke Constraint).cert] This certificate will trigger a 
        failure in the type checking because the registers typing for the Invoke 
        instruction is not adequate. In particular, the return value of the invoked
        method is $H$ while successor's $res$ is $L$.
      \item[InvokeConstraint\_typing] Show the policy of the method invoked, and the 
        typing for Invoke instruction and its successor's.
      \item[Certificate (mismatch instructions).cert] This certificate will trigger a
        failure in the type checking because the number of instructions in the actual
        DEX file and the certificate does not match.
      \item[Certificate (Move Constraint)] This certificate will trigger a failure in
        the type checking because the Move instruction is trying to move a value with
        high security level to the local variable which has low security level.
      \item[MoveConstraint\_typing] Show the policy of the method and also the typing
        for the Move instruction. 
      \item[Certificate (MoveResult rt).cert] This certificate will trigger a failure 
        in the type checking because the registers typing for the MoveResult 
        instruction $\nsqsubseteq$ its successor's.
      \item[MoveResultRT\_typing] Show the typing for the MoveResult instruction and 
        its successor
      \item[Certificate (Move rt).cert] This certificate will trigger a failure 
        in the type checking because the registers typing for the Move. 
        instruction $\nsqsubseteq$ its successor's.
      \item[MoveRT\_typing] Show the typing for the Move instruction and its successor
      \item[Certificate (Nop rt).cert] This certificate will trigger a failure 
        in the type checking because the registers typing for the CheckCast. 
        instruction $\nsqsubseteq$ its successor's. We label the instruction as Nop in
        this prototype because we have not fully implement the exception handling 
        mechanism, and as such we just treat CheckCast as another Nop instruction.
      \item[NopRT\_typing] Show the typing for the CheckCast instruction and 
        its successor
    \end{description}
  \item[files/Demo/] This folder contains the screen capture of the modification needed 
    to enable the flag no-optimize. This folder also contain the screen recordings of 
    the running examples of the certificates and how to translate the certificate 
    from JVM to DEX. As a note, these videos have been edited to shorten the idle 
    waiting.
    \begin{description}
      \item[DemoApp-TypeCheck.mp4] This is a screen recording of a successful type 
        checking on the actual Android phone.
      \item[DemoCert-ConstRT.mp4] This is a screen recording of a failed type checking
        corresponding to the file ``Certificate (Const rt).cert''
      \item[DemoCert-InvokeConstraint.mp4] This is a screen recording of a failed type  
        checking corresponding to the file ``Certificate (Invoke Constraint).cert''
      \item[DemoCert-MoveConstraint.mp4] This is a screen recording of a failed type 
        checking corresponding to the file ``Certificate (Move Constraint).cert''
      \item[DemoCert-MoveResultRT.mp4] This is a screen recording of a failed type 
        checking corresponding to the file ``Certificate (MoveResult rt).cert''
      \item[DemoCert-MoveRT.mp4] This is a screen recording of a failed type checking
        corresponding to the file ``Certificate (Move rt).cert''
      \item[DemoCert-NopRT.mp4] This is a screen recording of a failed type checking
        corresponding to the file ``Certificate (Nop rt).cert''
      \item[DemoCert-Setup.mp4] This is a screen recording of how to get the 
        intermediate Java classes, then feed it along with the policies and CDR for
        the JVM bytecode to the OCaml program to produce the DEX certificate. The 
        produced certificate is then bundled along with the test application.
      \item[Gradle1.png] This is a screen capture of modifying the gradle version in
        the Android Studio. We need to change the highlighted value into
        ``com.android.tools.build:gradle:1.3.1''
      \item[Gradle2.png] This is a screen capture of modifying the gradle setting of
        the app to enable the compilation with flag no-optimize.
    \end{description}
\end{description}

\end{document}
